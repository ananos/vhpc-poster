%\documentclass[portrait,final,a0paper]{baposter}
\documentclass[a1paper,portrait,final,fontscale=0.5]{baposter}
% Usa a4shrink for an a4 sized paper.

\tracingstats=2

\usepackage{calc}
\usepackage{graphicx}
\usepackage{amsmath}
\usepackage{amssymb}
\usepackage{relsize}
\usepackage{multirow}
\usepackage{bm}

\usepackage{graphicx}
\usepackage{multicol}

\usepackage{pgfbaselayers}
\pgfdeclarelayer{background}
\pgfdeclarelayer{foreground}
\pgfsetlayers{background,main,foreground}

\usepackage{times}
\usepackage{helvet}
%\usepackage{bookman}
\usepackage{palatino}

\newcommand{\captionfont}{\footnotesize}

\selectcolormodel{cmyk}

\graphicspath{{images/}}


%%%%%%%%%%%%%%%%%%%%%%%%%%%%%%%%%%%%%%%%%%%%%%%%%%%%%%%%%%%%%%%%%%%%%%%%%%%%%%%%
%%%% Some math symbols used in the text
%%%%%%%%%%%%%%%%%%%%%%%%%%%%%%%%%%%%%%%%%%%%%%%%%%%%%%%%%%%%%%%%%%%%%%%%%%%%%%%%
% Format
\newcommand{\Matrix}[1]{\begin{bmatrix} #1 \end{bmatrix}}
\newcommand{\Vector}[1]{\Matrix{#1}}
\newcommand*{\SET}[1]  {\ensuremath{\mathcal{#1}}}
\newcommand*{\MAT}[1]  {\ensuremath{\mathbf{#1}}}
\newcommand*{\VEC}[1]  {\ensuremath{\bm{#1}}}
\newcommand*{\CONST}[1]{\ensuremath{\mathit{#1}}}
\newcommand*{\norm}[1]{\mathopen\| #1 \mathclose\|}% use instead of $\|x\|$
\newcommand*{\abs}[1]{\mathopen| #1 \mathclose|}% use instead of $\|x\|$
\newcommand*{\absLR}[1]{\left| #1 \right|}% use instead of $\|x\|$

\def\norm#1{\mathopen\| #1 \mathclose\|}% use instead of $\|x\|$
\newcommand{\normLR}[1]{\left\| #1 \right\|}% use instead of $\|x\|$

%%%%%%%%%%%%%%%%%%%%%%%%%%%%%%%%%%%%%%%%%%%%%%%%%%%%%%%%%%%%%%%%%%%%%%%%%%%%%%%%
% Multicol Settings
%%%%%%%%%%%%%%%%%%%%%%%%%%%%%%%%%%%%%%%%%%%%%%%%%%%%%%%%%%%%%%%%%%%%%%%%%%%%%%%%
\setlength{\columnsep}{0.7em}
\setlength{\columnseprule}{0mm}


%%%%%%%%%%%%%%%%%%%%%%%%%%%%%%%%%%%%%%%%%%%%%%%%%%%%%%%%%%%%%%%%%%%%%%%%%%%%%%%%
% Save space in lists. Use this after the opening of the list
%%%%%%%%%%%%%%%%%%%%%%%%%%%%%%%%%%%%%%%%%%%%%%%%%%%%%%%%%%%%%%%%%%%%%%%%%%%%%%%%
\newcommand{\compresslist}{%
\setlength{\itemsep}{1pt}%
\setlength{\parskip}{0pt}%
\setlength{\parsep}{0pt}%
}


%%%%%%%%%%%%%%%%%%%%%%%%%%%%%%%%%%%%%%%%%%%%%%%%%%%%%%%%%%%%%%%%%%%%%%%%%%%%%%
%%% Begin of Document
%%%%%%%%%%%%%%%%%%%%%%%%%%%%%%%%%%%%%%%%%%%%%%%%%%%%%%%%%%%%%%%%%%%%%%%%%%%%%%

\begin{document}

%%%%%%%%%%%%%%%%%%%%%%%%%%%%%%%%%%%%%%%%%%%%%%%%%%%%%%%%%%%%%%%%%%%%%%%%%%%%%%
%%% Here starts the poster
%%%---------------------------------------------------------------------------
%%% Format it to your taste with the options
%%%%%%%%%%%%%%%%%%%%%%%%%%%%%%%%%%%%%%%%%%%%%%%%%%%%%%%%%%%%%%%%%%%%%%%%%%%%%%
% Define some colors
\definecolor{silver}{cmyk}{0,0,0,0.3}
%\definecolor{coral}{cmyk}{0,0.5,0.69,0.0}
\definecolor{coral}{cmyk}{0,0.58,0.69,0.0}
\definecolor{lightcoral}{cmyk}{0,0.48,0.59,0.0}
\definecolor{black}{cmyk}{0,0,0.0,1.0}
\definecolor{darkYellow}{cmyk}{0,0,1.0,0.5}
\definecolor{darkSilver}{cmyk}{0,0,0,0.1}
\definecolor{darksalmon}{cmyk}{0,0.36,0.48,0.09}
\definecolor{salmon5}{cmyk}{0,0.41,0.41,0.56}

\definecolor{lightercoral}{cmyk}{0,0.25,0.25,0}
%\definecolor{lightercoral}{cmyk}{0,0.40,0.47,0.08}
\definecolor{apricot}{cmyk}{0.0,0.36,0.57,0.02}
\definecolor{lightestapricot}{cmyk}{0,0,0.05,0.0}

%%
\typeout{Poster Starts}
\background{
  \begin{tikzpicture}[remember picture,overlay]%
    %\draw (current page.north west)+(-2em,2em) node[anchor=north west] {\includegraphics[height=1.1\textheight]{silhouettes_background}};

  \end{tikzpicture}%
}

\newlength{\leftimgwidth}
\begin{poster}%
  % Poster Options
  {
  % Show grid to help with alignment
  grid=no,
  % Column spacing
  colspacing=1em,
  % Color style
%  bgColorOne=white,
%  bgColorOne=darksalmon,
  bgColorOne=white,
  borderColor=blue,
  %headerColorOne=salmon5,
  headerColorOne=white,
  headerColorTwo=blue,
  %headerFontColor=lightcoral,
  %boxColorOne=lightercoral,
  boxColorOne=white,
  boxColorTwo=lightercoral,
  % Format of textbox
  textborder=roundedleft,
%  textborder=rectangle,
  % Format of text header
  eyecatcher=no,
  headerborder=open,
  headerheight=0.12\textheight,
  headershape=rounded,
  headershade=shade-tb,
  headerfont=\Large\textsf, %Sans Serif
  boxshade=plain,
%  background=shade-tb,
  background=plain,
  linewidth=2pt
  }
  % Eye Catcher
  {\includegraphics[width=10em]{D1077}} % No eye catcher for this poster. (eyecatcher=no above). If an eye catcher is present, the title is centered between eye-catcher and logo.
  % Title
  {\sf %Sans Serif
  \bf% Serif
  \vspace{2em}
  \begin{center} 8th Workshop on Virtualization in High-Performance \\Cloud Computing (VHPC'13)\end{center}}
  % Authors
  {\sf %Sans Serif
  }
  % University logo
  {% The makebox allows the title to flow into the logo, this is a hack because of the L shaped logo.
    \makebox[8em][r]{%
      \begin{minipage}{16em}
        \hfill
      %  \includegraphics[height=2em]{cslab_logo.pdf}
      %  \includegraphics[height=7.0em]{ntua_logo.pdf}
      \end{minipage}
    }
  }

  \tikzstyle{light shaded}=[top color=baposterBGtwo!30!white,bottom color=baposterBGone!30!white,shading=axis,shading angle=30]

  % Width of left inset image
     \setlength{\leftimgwidth}{0.78em+8.0em}

    % A coloured circle useful as a bullet with an adjustably strong filling
    \newcommand{\colouredcircle}[1]{%
      \tikz{\useasboundingbox (-0.2em,-0.32em) rectangle(0.2em,0.32em); \draw[draw=black,fill=baposterBGone!80!black!#1!white,line width=0.03em] (0,0) circle(0.18em);}}

  \headerbox{Topics of Interest}{name=contribution,column=0,span=1.5}{
%%%%%%%%%%%%%%%%%%%%%%%%%%%%%%%%%%%%%%%%%%%%%%%%%%%%%%%%%%%%%%%%%%%%%%%%%%%%%%
%Virtualization has become a common abstraction layer in modern
%data centers, enabling resource owners to manage complex
%infrastructure independently of their applications. Conjointly,
%virtualization is becoming a driving technology for a manifold of
%industry grade IT services. The cloud concept includes the notion
%of a separation between resource owners and users, adding services
%such as hosted application frameworks and queueing. Utilizing the
%same infrastructure, clouds carry significant potential for use in
%high-performance scientific computing. The ability of clouds to provide
%for requests and releases of vast computing resources dynamically and
%close to the marginal cost of providing the services is unprecedented in
%the history of scientific and commercial computing.
%
%This workshop aims to bring together industrial providers with the
%scientific community in order to foster discussion, collaboration
%and mutual exchange of knowledge and experience.
%
%The workshop will be one day in length, composed of 20 min
%paper presentations, each followed by 10 min discussion sections.
%Lightning talks are limited to 5 minutes. Presentations may be
%accompanied by interactive demonstrations.
%
Topics of interest include, but are not limited to:

\begin{itemize}
\item Management, deployment and monitoring of VM-based environments
\item VM-cloud performance monitoring
\item VM cloud topology management and optimization
\item Operating systems virtualization supportpptimization
\item VM-based cloud performance modelling
\item Network virtualization for VM-environments
\item Data virtualization
\item Cloudbursting
\item Evolved grid architectures including such based on network virtualization
\item Workload characterization for VM-based environments
\item Optimized communication libraries/protocols in the cloud
\item System and process/bytecode VM convergence
\item Cloud frameworks and APIs
\item GPU Virtualization architectures and APIs
\item Checkpointing/migration of large compute jobs
\item Instrumentation interfaces and languages
\item VMM performance (auto-)tuning on various load types
\item Cloud reliability, fault-tolerance, and security
\item Heterogeneous virtualized environments
\item Paravirtualized I/O
\item Services in cloud HPC
\item Research and education use cases
\item Virtualization in cloud, cluster and grid environments
\item Cross-layer VM optimizations
\item Cloud HPC use cases including optimizations
\item Energy-aware virtualization
\item Performance and cost modelling
\item QoS and and service levels
\item Languages for describing highly-distributed compute jobs
\item VM cloud and cluster distribution algorithms, load balancing
\item Instrumentation interfaces and languages
\item Hypervisor extensions and tools for cluster and grid computing
\item Virtual machine monitor platforms
\item Cluster provisioning in the cloud
\end{itemize}
  }
  
  

%%%%%%%%%%%%%%%%%%%%%%%%%%%%%%%%%%%%%%%%%%%%%%%%%%%%%%%%%%%%%%%%%%%%%%%%%%%%%%
  \headerbox{Important dates}{name=motivation,column=0,span=1.5, below=contribution}{
%%%%%%%%%%%%%%%%%%%%%%%%%%%%%%%%%%%%%%%%%%%%%%%%%%%%%%%%%%%%%%%%%%%%%%%%%%%%%%

Rolling abstract submission
\begin{itemize}
\item Full paper submission: \textbf{Sept. 23rd}, 2013
\item Acceptance notification: \textbf{Oct. 21st}, 2013
\item Camera-ready version due: \textbf{Nov. 8th}, 2013
\item Lightning talk abstracts: \textbf{Aug. 9th}, 2013
\item Lightning talk notification: \textbf{Sep. 22nd}, 2013
\item Workshop date: \textbf{Nov. 22nd}, 2013
\end{itemize}
}
%%%%%%%%%%%%%%%%%%%%%%%%%%%%%%%%%%%%%%%%%%%%%%%%%%%%%%%%%%%%%%%%%%%%%%%%%%%%%%
  \headerbox{}{name=questions,row=0,column=1.5,span=1.5}{
%%%%%%%%%%%%%%%%%%%%%%%%%%%%%%%%%%%%%%%%%%%%%%%%%%%%%%%%%%%%%%%%%%%%%%%%%%%%%%
%\begin{figure}
%\centering
%\includegraphics[width=\linewidth]{logo_acm_icps.png}
%\includegraphics[width=\linewidth]{sc13_logo.png}
%\includegraphics[width=.2\linewidth]{ntua_logo.pdf}
%\includegraphics[width=\linewidth]{cslab_logo.pdf}
\begin{center}
\includegraphics[width=.58\linewidth]{logos.png}
\end{center}
  \vspace{4.5em}
  }


\end{poster}

\end{document}
